%%% Copyright (c) 2010, Илья w-495 Никитин
%%%
%%% Разрешается повторное распространение и использование как в виде исходного
%%% кода, так и в двоичной форме, если таковая будет получена, 
%%% с изменениями или без, при соблюдении следующих условий:
%%%
%%%     * При повторном распространении исходного кода должно оставаться
%%%       указанное выше уведомление об авторском праве, этот список условий и
%%%       последующий отказ от гарантий.
%%%     * Ни имя w-495, ни имена друзей или консультантов не могут быть
%%%       использованы в качестве поддержки или продвижения продуктов,
%%%       основанных на этом коде без предварительного письменного разрешения. 
%%%
%%% Этот код предоставлен владельцом авторских прав и/или другими
%%% сторонами "как она есть" без какого-либо вида гарантий, выраженных явно
%%% или подразумеваемых, включая, но не ограничиваясь ими, подразумеваемые
%%% гарантии коммерческой ценности и пригодности для конкретной цели. Ни в
%%% коем случае, если не требуется соответствующим законом, или не установлено
%%% в устной форме, ни один владелец авторских прав и ни одно  другое лицо,
%%% которое может изменять и/или повторно распространять программу, как было
%%% сказано выше, не несёт ответственности, включая любые общие, случайные,
%%% специальные или последовавшие убытки, вследствие использования или
%%% невозможности использования программы (включая, но не ограничиваясь
%%% потерей данных, или данными, ставшими неправильными, или потерями
%%% принесенными из-за вас или третьих лиц, или отказом программы работать
%%% совместно с другими программами), даже если такой владелец или другое
%%% лицо были извещены о возможности таких убытков.


\documentclass[unicode, 12pt, a4paper,oneside,fleqn]{article}
	%% Варианты []:
		% fleqn --- сдвигает формулы влево

	%% Варианты {}:
		% book
		% report
		% article
		% letter
		% minimal (???)

\usepackage{styles/main} 
	% подключаем набор стилей 

\ifpdf
	\hypersetup{ 
			pdffitwindow=false,
			pdfstartview={FitH},			
		pdftitle={Это шаблонный документ v0.92}, 
		pdfauthor={Илья w-495 Никитин}, 
		pdfcreator={LaTeX2e + TexMakerX}, 
		pdfsubject={Тема}, 
		pdfproducer={Илья w-495 Никитин}, 
		pdfkeywords={Шаблон}
	}
\fi

\newtheorem{definition}{Определение}

\begin{document}

%%%%%%%%%%%%%%%%%%%%%%%%%%%%%%%%%%%%%%%%%%%%%%%%%%%%%%%%%%%%%%%%%%%%%%%%%%%%%%%%
%%%
%%% бесполезное содержимое
%%%

	\begin{titlepage}
\begin{center} %% ПО ЦЕНТРУ

\bfseries
%%%%%%%%%%%%%%%%%%%%%%%%%%%%%%%%%%%%%%%%%%%%%%%%%%%%%%%%%%%%%%%%%%%%%%%%%%%%%%%%
%%%
%%% ВУЗ
%%%

	{\Large Высшая школа экономики
	} %% или что-то в этом духе

\vspace{48pt}

%%%%%%%%%%%%%%%%%%%%%%%%%%%%%%%%%%%%%%%%%%%%%%%%%%%%%%%%%%%%%%%%%%%%%%%%%%%%%%%%
%%%
%%% Факультет
%%%

	{\large Факультет компьютерных наук
	}

	%{\large Факультет иностранных языков
	%
	%}

\vspace{36pt}
%%%%%%%%%%%%%%%%%%%%%%%%%%%%%%%%%%%%%%%%%%%%%%%%%%%%%%%%%%%%%%%%%%%%%%%%%%%%%%%%
%%%
%%% Кафедра
%%%

	{\large Направление \enquote{Прикладная математика и информатика}
	
	} %% или что-то в этом духе

\vspace{48pt}
%%%%%%%%%%%%%%%%%%%%%%%%%%%%%%%%%%%%%%%%%%%%%%%%%%%%%%%%%%%%%%%%%%%%%%%%%%%%%%%%
%%%
%%% Класс работы
%%%

	% Шаблон по курсу \enquote{Какой-то предмет} 
	% Лекции по курсу \enquote{Какой-то предмет} 
	% Лабораторная работа по курсу \enquote{Какой-то предмет} 
	 Курсовая работа
	% Курсовой проект по курсу \enquote{Какой-то предмет} 

\vspace{12pt}
%%%%%%%%%%%%%%%%%%%%%%%%%%%%%%%%%%%%%%%%%%%%%%%%%%%%%%%%%%%%%%%%%%%%%%%%%%%%%%%%
%%%
%%% Название работы
%%%

	{\Large Обучение формуле Хорна с запросами к несовершенному оракулу.
	}

\end{center} %% УЖЕ НЕ ПО ЦЕНТРУ

\vspace{60pt}
%%%%%%%%%%%%%%%%%%%%%%%%%%%%%%%%%%%%%%%%%%%%%%%%%%%%%%%%%%%%%%%%%%%%%%%%%%%%%%%%
%%%
%%% Автор(ы)
%%%

	\begin{flushright}
		\begin{tabular}{rl}
			Студент: & А.\,А. Урусов \\
			Преподаватель: & С.\,А. Объедков \\
		\end{tabular}
	\end{flushright}

\vfill
%%%%%%%%%%%%%%%%%%%%%%%%%%%%%%%%%%%%%%%%%%%%%%%%%%%%%%%%%%%%%%%%%%%%%%%%%%%%%%%%
%%%
%%% Дата
%%%

	\begin{center} %% ПО ЦЕНТРУ
		\bfseries
		Москва, 2017
	\end{center}
	
\end{titlepage} 

 	% титульный лист
	\pagebreak
	\section{Постановка задачи.}

Рассмотрим сначала оригинальную версию задачи. Пусть есть некоторый набор бинарных переменных переменных $x_1, \dots, x_n$.

\begin{definition}
	Условие Хорна - это дизъюнкция литералов (то есть переменных или их отрицаний), в которой не более одного раза встречается переменная без отрицания. Таким образом, оно принимает вид $x_{i_1} \land x_{i_2} \land \dots x_{i_k} \rightarrow a$, где $a$ - либо одна из переменных, либо константа True или False.
\end{definition}

\begin{definition}
	Формула Хорна - бинарное выражение вида $L_1 \land L_2 \land \dots \land L_k$, где $L_i$ - некоторое условие Хорна.
\end{definition}

Пусть задана также некоторая формула Хорна $H*$ на этих переменных. Мы можем взаимодействовать с ней посредством оракула, умеющего отвечать на 2 типа запросов:

\begin{itemize}
	\item Запрос принадлежности. По заданным значениям переменных определить значение $H*$ на этих значениях.
	\item Запрос эквивалентности. По заданной формуле Хорна $H$ определить, является ли она эквивалентной $H*$. При этом если ответ отрицательный, то также возвращается контрпример $x$, который может быть положительным ($H(x) = 0, H*(x) = 1$) или отрицательным ($H(x) = 1, H*(x) = 0$).
\end{itemize}

Цель задачи - научиться вычислять $H*$.

Заметим, что если ответ на первый запрос достаточно очевиден (нужно просто подставить в $H*$ заданные значения переменных), то на запрос эквивалентности отвечать довольно сложно (хотя и возможно за полиномиальное время). Поэтому имеет смысл рассмотреть несовершенный оракул, то есть такой, который отвечает только на запросы первого типа.
	% Описание и, возможно, доказательство оригинального алгоритма Англуин
\section{Алгоритм решения задачи с совершенным оракулом.}

\emph{Алгоритм также описан в [1] и [3].}

Предположим сначала, оракул все-таки умеет отвечать на запросы второго типа. Рассмотрим следующий алгоритм нахождения множества импликаций.

Пусть изначально множество импликаций $\mathcal{L} = \emptyset$. Далее, если на очередном шаге наше множество импликаций эквивалентно множеству объектов, то можно просто вернуть его. Иначе, есть некоторый контрпример $X$. Далее возможны 2 случая:

\begin{itemize}
	\item $X$ не удовлетворяет $\mathcal{L}$. Тогда этот контрпример положительный. В этом случае для каждой импликации $(A \rightarrow B) \in \mathcal{L}$ если $A \subset X$, то заменим эту импликацию на $(A \rightarrow B \cap X)$. Таким образом, мы изменяем следствие каждой импликации так, чтобы контрпример ей удовлетворял.
	\item $X$ удовлетворяет $\mathcal{L}$, то есть контрпример отрицательный. В этом случае мы находим такую импликацию $A \rightarrow B$, что $С = X \cap A \neq A$, и при этом $С$ не входит в множества (последнее проверяется запросом к оракулу), и заменяем ее на $C \rightarrow B$. Если же такой импликации не нашлось, то просто добавим в множество импликацию $X \rightarrow M$. Таким образом, мы модифицируем множество $\mathcal{L}$ так, что ему не будет удовлетворять $X$.
\end{itemize}

Эту последовательность действий будем повторять до тех пор, пока не получим множество импликаций, эквивалентное заданному (это мы проверяем в начале каждого шага). В [3] доказано, что этот алгоритм не только завершается корректно, но и работает полиномиальное время, а именно, ассимптотика этого алгоритма $\tilde{O}(m^2n^2)$, при этом делая $O(mn)$ запросов эквивалентности и $O(m^2n)$ запросов принадлежности, где $m$ - количество импликаций в $H*$, $n$ - количество переменных.
	% Описание идеи моделирования запроса эквивалентности через запрос принадлежности + параметры алгоритма
\section{Модификация алгоритма для несовершенного оракула.}

Вернемся к несовершенному оракулу. Как было сказано выше, он отличается тем, что не допускает запросов на эквивалентность.

В [3] упоминается о невозможности обучения формуле Хорна за полиномиальное время в таких условиях. Однако, мы можем попробовать обучиться формуле Хорна неточно, но с относительно небольшой потерей качества. Для этого смоделируем запросы на эквивалентность следующим образом: на очередном шаге зададим количество итераций $iter$, зависящее от номера шага алгоритма, описанного в предыдущей секции, а также параметров алгоритма, и сгенерируем $iter$ случайных комбинаций значений переменных. Для каждой комбинации $x = (x_1, \dots, x_n)$ проверим, является ли она контрпримером. Для этого нужно проверить, удовлетворяет ли $x$ импликациям из $H$ (для этого просто пройдем по этим импликациям) и импликациям из $H*$ (для этого достаточно выполнить запрос на принадлежность). Если за заданное количество итераций мы нашли контрпример, то возвращаем его, если не нашли, то считаем, что примера не существует, и заданное множество импликаций эквивалентно.

Осталось понять, как выбрать количество итераций. Зададим сначала 2 параметра $0 < \epsilon, \delta < 1$. Тогда, согласно [2], если выбрать $iter = \left\lceil \frac{1}{\epsilon}\left(i + \ln \frac{1}{\delta}\right) \right\rceil$, то имеют место быть следующие гарантии на качество. Обозначим как Mod(H) множество комбинаций значений переменных, которые удовлетворяют импликациям $H$. Тогда алгоритм Англуин, описанный в предыдущей секции, с приведенной рандомизированной реализацией запроса эквивалентности гарантирует, что:

\begin{equation}
	Pr\left(\frac{\left| Mod(H) \triangle Mod(H*) \right|}{2^n} \leq \epsilon \right) \geq 1 - \delta
\end{equation}
	% Описание задачи на основании данных: построить импликации по данным, + модификации алгоритма под эту задачу
\section{Обучение импликациям на основе данных}
\subsection{Постановка задачи}
В реальной жизни в качестве оракула обычно выступает человек или группа людей, обладающих экспертными знаниями в некоторой области. Однако, во многих случаях доступа к таким людям может не быть по ряду причин. Например, мы можем не обладать полными знаниями о некоторой области, или нахождение этих знаний является слишком долгим в связи с малым количеством экспертов. Однако, мы можем иметь некоторые данные о примерах объектов, о которых пытаемся что-то выучить. В связи с этим появляется задача обучения по выборке. Формально, пусть у нас есть набор объектов $X$, у каждого из которых есть некоторое количество бинарных признаков.

\begin{definition}
Замыкание $X$ - это множество $a = x_1 \land x_2 \land \dots \land x_k, x_i \in X$, то есть множество всевозможных пересечений объектов из $X$.
\end{definition}

Тогда наша задача состоит в нахождении множества импликаций, которому удовлетворяют в точности все объекты из замыкания.

Чтобы применить уже известный алгоритм к такой постановке задачи, нужно описать, как выполняются запросы принадлежности (отметим, что запросы эквивалентности мы уже умеем моделировать через запросы принадлежности, соответственно, их не требуется моделировать снова).

По сути запрос принадлежности заключается в проверке того, что заданный объект $x$ является пересечением некоторого подмножества $X'$ имеющихся объектов. Заметим, что все элементы из $X'$ должны иметь все признаки, которые есть у $x$. Тогда возьмем множество $X''$ всех объектов, содержащих все признаки $x$. Тогда пересечение всех элементов $X''$, очевидно, содержит все признаки из $x$. Кроме того, если существует множество $X'$, пересечение элементов которого равно $x$, то пересечение элементов $X'' \supseteq X'$ тоже равно $x$. Значит, достаточно проверить, что пересечение всех множеств, содержащих $x$, равно $x$.

\subsection{Оптимизации алгоритма}
\subsubsection{Ограничение по времени}
Поскольку теоретическая оценка на качество довольно грубая, можно предположить, что на практике заданное качество достигается значительно быстрее, чем в момент сходимости. Тогда логично попробовать чем-то ограничить время рандомизированному алгоритму. Это можно сделать двумы способами. Во-первых, можно просто задать некоторый константный порог. Это имеет смысл, поскольку часто алгоритм работает достаточно медленно. Кроме того, для задачи построения импликаций по данным уже существуют точные алгоритмы, которые подробно описаны в [2]. Поэтому логично попробовать ограничить рандомизированный алгоритм некоторой долей времени работы какого-нибудь из них.

\subsection{Использование данных для более точного корректирования импликаций}
\emph{Coming soon}

\subsection{Использование замыкания}
Напомним, что в рандомизированном оракуле мы генерируем случайные объекты, а затем проверяем, являются ли они контрпримерами. Заметим, что при условии использования оптимизации из предыдущего пункта мы уверены, что если контрпример есть, то он отрицательный. Значит, он точно должен удовлетворять нашим импликациям. Тем не менее, многие из сгенерированных объектов не будут удовлетворять этим свойством. Чтобы исправить это, можно замыкать каждый генерируемый объект относительно нашего текущего множества импликаций. Тогда полученный объект достаточно будет проверить на принадлежность соответствующим запросом к оракулу. Тем не менее, замыкание - это довольно сложная операция, поэтому следует ожидать ухудшения производительности (или качество - если мы ограничиваем время).
	\section{Тестирование}

Для тестирования удобно использовать как раз постановку задачи на основе данных. При этом можно использовать как сгенерированные модельные примеры, так и реальные данные. Начнем с первых. Будем генерировать выборку случайно. При этом будем задавать количество признаков и объектов, а также плотность, то есть вероятность появления отдельно заданного признака у заданного объекта. Кроме того, мы можем задавать параметры нашего алгоритма, а именно, $\epsilon, \delta$, а также использовать различные варианты оптимизаций.

Теперь нужно ввести метрики качества. Во-первых, логично измерять качество работы алгоритма, поскольку он не является точным. В качестве метрики качества можно использовать ту величину, для которой мы гарантируем теоретическую оценку. Поскольку объектов экспоненциально много, честный расчет этой величины занимает очень много времени, поэтому будем использовать рандомизированную оценку этой величины: возьмем 10000 случайных объектов, проверим, сколько из них входят в числитель, и усредним размером выборки.

Теперь вернемся к тому, зачем мы вообще хотим обучаться импликациям. В большой степени это необходимо с целью имитации оператора замыкания. Это означает, что для произвольного набора признаков мы хотим понимать, какими еще признаками обладает объект, обладающий этими. Тогда естественной метрикой будет доля объектов, для которых мы правильно определяем замыкание. Ее тоже можно считать рандомизированно: возьмем выборку из 1000 случайных объектов и проверим, для скольких из них замыкание на основе наших импликаций совпадает с замыканием на основе реальных импликаций, и также усредним размером выборки.

Теперь разберемся с параметрами алгоритма. В первую очередь посмотрим на оптимизации. Eсть 3 основных версии алгоритма:

\begin{enumerate}
	\item Базовый алгоритм. Эта версия не использует никаких оптимизаций, кроме, возможно, ограничения по времени.
	\item Улучшенный алгоритм. Эта версия использует оптимизацию 2 из предыдущей секции для задачи на основе данных.
	\item Улучшенный алгоритм с улучшенным оракулом. Эта версия использует обе оптимизации 2 и 3 из предыдущей секции. Отметим, что оптимизация 3 частично основана на гарантиях, предоставляемых оптимизацией 2, и, соответственно, имеет мало смысла без нее.
\end{enumerate}

Кроме того, используем первую оптимизацию следующим образом: возьмем из [2] оптимизированный алгоритм Гантера с использованием линейного замыкания, и ограничим половиной времени его работы наш алгоритм.

Посмотрим на зависимости качества работы алгоритмов от параметров выборки. Ниже приведены графики зависимости 2х метрик, о которых говорилось выше, от одного из параметров выборки при фиксированных двух оставшихся. Здесь guaranteed metric - это метрика, теоретическая оценка на которую гарантируется в оригинальной версии алгоритма, а quality - доля ошибок оператора замыкания.

\includegraphics[scale=0.3]{img/obj-accuracy}
\includegraphics[scale=0.3]{img/obj-quality}

\includegraphics[scale=0.3]{img/attr-accuracy}
\includegraphics[scale=0.3]{img/attr-quality}

\includegraphics[scale=0.3]{img/dens-accuracy}
\includegraphics[scale=0.3]{img/dens-quality}

Как можно видеть, лучшим качеством обладает оригинальный алгоритм. Это, вероятно, происходит потому, что оптимизации хоть и потенциально улучшают качество, значительно ухудшают производительность алгоритма, и поэтому в условиях ограниченного времени не успевают достичь требуемого качества.

\emph{Coming soon}
% Эксперименты для оригинального алгоритма
% Описание оптимизации с getconsequence
% Эксперименты с getconsequence
% Описание оптимизации с closure
% Эксперименты с closure
% Описание оптимизации "обрезать время"
% Эксперименты с обрезанием времени
% Эксперименты: время работы на реальных данных
	\newpage
\section{Литература}

[1] K. Bazhanov, S. Obiedkov Optimizations in computing the Duquenne–Guigues basis of implications

[2] S. Obiedkov Probabilistic Computation of the Canonical Basis with Membership Queries

[3] D. Angluin, M. Frazier, L. Pitt Learning Conjunctions of Horn Clauses

[4] H. Kautz, M. Kearns, B. Selman Horn approximations of empirical data

	%%%%%%%%%%%%%%%%%%%%%%%%%%%%%%%%%%%%%%%%%%%%%%%%%%%%%%%%%%%%%%%%%%%%%%%%%%%%%%%%
	%%%
	%%% дополнительное (свое) задание верхнего колонтитула
	%%% 
	%%%
	%	\makeatletter
	%	\renewcommand{\@oddhead}{ \textcolor{blue}{Лекция (задача) \arabic{lections}} \hfil \par
	%	\hfil  \leftmark \hfil \rightmark }
	%	\makeatother
%& -shell-escape
	
%%%%%%%%%%%%%%%%%%%%%%%%%%%%%%%%%%%%%%%%%%%%%%%%%%%%%%%%%%%%%%%%%%%%%%%%%%%%%%%%
%%%
%%% полезное содержимое
%%%

	% пример %%%%%%%%%%%%%%%%%%%%%%%%%%%%%%%%%%%%%%%%%%%%%%%%%%%%%%%%%%%%%%%%%%%
	
		%\input{\SRC/-example-about}
		%\input{\SRC/-example-sources}
		%\input{\SRC/-example-pictures}
		%\input{\SRC/-example-plot}
		%\input{\SRC/-example-text}
	
	% лекции %%%%%%%%%%%%%%%%%%%%%%%%%%%%%%%%%%%%%%%%%%%%%%%%%%%%%%%%%%%%%%%%%%%
	
	%	\input{\SRC/lct-01} %% лекция #1
	
	% лабы\курсовые %%%%%%%%%%%%%%%%%%%%%%%%%%%%%%%%%%%%%%%%%%%%%%%%%%%%%%%%%%%%%%%%%%%
	
	%	\input{\SRC/work-problem} 		%% постановка
	%	\input{\SRC/work-theory} 		%% теоретическая часть
	%	\input{\SRC/work-solution} 		%% решение
	%	\input{\SRC/work-example} 		%% примеры
	%	\input{\SRC/work-conclusions} 	%% выводы
		
\end{document}

%%
%%
%%

