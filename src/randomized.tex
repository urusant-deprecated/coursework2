% Описание идеи моделирования запроса эквивалентности через запрос принадлежности + параметры алгоритма
\section{Модификация алгоритма для несовершенного оракула.}

Вернемся к несовершенному оракулу. Как было сказано выше, он отличается тем, что не допускает запросов на эквивалентность.

В [3] упоминается о невозможности обучения формуле Хорна за полиномиальное время в таких условиях. Однако, мы можем попробовать обучиться формуле Хорна неточно, но с относительно небольшой потерей качества. Для этого смоделируем запросы на эквивалентность следующим образом: на очередном шаге зададим количество итераций $iter$, зависящее от номера шага алгоритма, описанного в предыдущей секции, а также параметров алгоритма, и сгенерируем $iter$ случайных комбинаций значений переменных. Для каждой комбинации $x = (x_1, \dots, x_n)$ проверим, является ли она контрпримером. Для этого нужно проверить, удовлетворяет ли $x$ импликациям из $H$ (для этого просто пройдем по этим импликациям) и импликациям из $H*$ (для этого достаточно выполнить запрос на принадлежность). Если за заданное количество итераций мы нашли контрпример, то возвращаем его, если не нашли, то считаем, что примера не существует, и заданное множество импликаций эквивалентно.

Осталось понять, как выбрать количество итераций. Зададим сначала 2 параметра $0 < \epsilon, \delta < 1$. Тогда, согласно [2], если выбрать $iter = \left\lceil \frac{1}{\epsilon}\left(i + \ln \frac{1}{\delta}\right) \right\rceil$, то имеют место быть следующие гарантии на качество. Обозначим как Mod(H) множество комбинаций значений переменных, которые удовлетворяют импликациям $H$. Тогда алгоритм Англуин, описанный в предыдущей секции, с приведенной рандомизированной реализацией запроса эквивалентности гарантирует, что:

\begin{equation}
	Pr\left(\frac{\left| Mod(H) \triangle Mod(H*) \right|}{2^n} \leq \epsilon \right) \geq 1 - \delta
\end{equation}