% Описание и, возможно, доказательство оригинального алгоритма Англуин
\section{Алгоритм решения задачи с совершенным оракулом.}

\emph{Алгоритм также описан в [1] и [3].}

Предположим сначала, оракул все-таки умеет отвечать на запросы второго типа. Рассмотрим следующий алгоритм нахождения множества импликаций.

Пусть изначально множество импликаций $\mathcal{L} = \emptyset$. Далее, если на очередном шаге наше множество импликаций эквивалентно множеству объектов, то можно просто вернуть его. Иначе, есть некоторый контрпример $X$. Далее возможны 2 случая:

\begin{itemize}
	\item $X$ не удовлетворяет $\mathcal{L}$. Тогда этот контрпример положительный. В этом случае для каждой импликации $(A \rightarrow B) \in \mathcal{L}$ если $A \subset X$, то заменим эту импликацию на $(A \rightarrow B \cap X)$. Таким образом, мы изменяем следствие каждой импликации так, чтобы контрпример ей удовлетворял.
	\item $X$ удовлетворяет $\mathcal{L}$, то есть контрпример отрицательный. В этом случае мы находим такую импликацию $A \rightarrow B$, что $С = X \cap A \neq A$, и при этом $С$ не входит в множества (последнее проверяется запросом к оракулу), и заменяем ее на $C \rightarrow B$. Если же такой импликации не нашлось, то просто добавим в множество импликацию $X \rightarrow M$. Таким образом, мы модифицируем множество $\mathcal{L}$ так, что ему не будет удовлетворять $X$.
\end{itemize}

Эту последовательность действий будем повторять до тех пор, пока не получим множество импликаций, эквивалентное заданному (это мы проверяем в начале каждого шага). В [3] доказано, что этот алгоритм не только завершается корректно, но и работает полиномиальное время, а именно, ассимптотика этого алгоритма $\tilde{O}(m^2n^2)$, при этом делая $O(mn)$ запросов эквивалентности и $O(m^2n)$ запросов принадлежности, где $m$ - количество импликаций в $H*$, $n$ - количество переменных.