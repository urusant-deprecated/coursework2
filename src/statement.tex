\section{Постановка задачи.}

Рассмотрим сначала оригинальную версию задачи. Пусть есть некоторый набор бинарных переменных переменных $x_1, \dots, x_n$.

\begin{definition}
	Условие Хорна - это дизъюнкция литералов (то есть переменных или их отрицаний), в которой не более одного раза встречается переменная без отрицания. Таким образом, оно принимает вид $x_{i_1} \land x_{i_2} \land \dots x_{i_k} \rightarrow a$, где $a$ - либо одна из переменных, либо константа True или False.
\end{definition}

\begin{definition}
	Формула Хорна - бинарное выражение вида $L_1 \land L_2 \land \dots \land L_k$, где $L_i$ - некоторое условие Хорна.
\end{definition}

Пусть задана также некоторая формула Хорна $H*$ на этих переменных. Мы можем взаимодействовать с ней посредством оракула, умеющего отвечать на 2 типа запросов:

\begin{itemize}
	\item Запрос принадлежности. По заданным значениям переменных определить значение $H*$ на этих значениях.
	\item Запрос эквивалентности. По заданной формуле Хорна $H$ определить, является ли она эквивалентной $H*$. При этом если ответ отрицательный, то также возвращается контрпример $x$, который может быть положительным ($H(x) = 0, H*(x) = 1$) или отрицательным ($H(x) = 1, H*(x) = 0$).
\end{itemize}

Цель задачи - научиться вычислять $H*$.

Заметим, что если ответ на первый запрос достаточно очевиден (нужно просто подставить в $H*$ заданные значения переменных), то на запрос эквивалентности отвечать довольно сложно (хотя и возможно за полиномиальное время). Поэтому имеет смысл рассмотреть несовершенный оракул, то есть такой, который отвечает только на запросы первого типа.